\section{Design goals}
\subsection{Criteria}\label{ssec:criteria}
We offer our system open source and free of charge online\footnote{https://github.com/Meijuh/psd}. We assume the application is a highly
popular one and is used among many people. We identify the most important quality criteria for this
environment as \emph{extensibility} and \emph{reusability}. The most affected stakeholders are the
\emph{developers}, \emph{maintainers}, \emph{testers} and \emph{users}.

Extensibility is an external criteria; it is most important for the user. Reusability is more
important for developers etc. Both criteria are also covered under the term modularity. A key
feature of bohnanza is to add extensions to the standard game. Modularity allow the user to easily use one of the many extensions, such
as High Bohn or spin-offs, such as Al Cabohne. Also, for developers modularity allows for easy implementations of these extensions.

A natural way of decomposing the application into modules to achieve modularity is to provide one base module which contains basic bohnanza
logic. Another module containing the std bohnanza game and the third module containing the high bohn extension. Using these modules allows us to
compile and test them separately. Also, to make an extension for bohnanza we can easily extend the base module. Every module is a distinct Eclipse
project.
 
\subsection{Flexibility}\label{ssec:flexibility}
Creating flexible software is not easy. Flexibility may introduce risks in software, but it allows for faster adaptation of software when
external changes are desired. The basic bohnanza module should allow for creating an extension easily. However one should not over-engineer
software to much. That is, by trying to anticipate what a future release of the software should support. The way we support flexibility in
the bohnanza game is only for creating extensions of the standard bohnanza game. These extensions can be created by subtyping certain
classes in the base module. An alternative would be to supply hooks in our base module. But one can only
guess if this approach works well with extensions we have not implemented, such as el Cabohne. 

One can not simply override any method if behaviour needs to be changed, because this is bad practice. As
Checkstyle\footnote{http://checkstyle.sourceforge.net/} suggests a method should either be final or abstract. Therefore if one needs to
override a method a final method should be made abstract and (parts of it) should be implemented in a subtype. We find this closely
relates to the \emph{open-closed principle} of modularity. The base module is \emph{open} because
no classes are final and can thus be extended. But it is \emph{closed} -- thus harder to extend --
because methods are final. In our base module behaviour is hard to \emph{modify} but behaviour can however
be easily \emph{added}, since classes are not final. The trade-off here is that it is harder to modify behaviour of the base module and thus
one can not easily break the way we intent bohnanza to work. On the other hand, it is unforeseeable if for some extension the base module
needs to be changed. If this is the case one has to change the accessibility modifiers described earlier.
