\section{Design goals}
\subsection{Criteria}\label{ssec:criteria}
The importance of particular quality criteria depend on in which environment the application is
used. This environment also influences the stakeholders; not every category stakeholder is affected
by the software system. We offer our system open sourced and free of charge
online.\footnote{https://github.com/Meijuh/psd}. We assume the application is a highly popular one
and is used among many people. We identify the most important quality criteria for this
environment as \emph{extensibility} and \emph{reusability}. The most affected stakeholders are the
\emph{developers}, \emph{maintainers}, \emph{testers} and \emph{users}.

Extensibility is an external criteria; it is most important for the user. Reusability is more
important for developers etc. Both criteria are also covered under the term modularity. A key
feature of bohnanza is to add extensions to the \gls{std} game. Modularity allow the user to easily use one of the many extensions, such
as High Bohn or spin-offs, such as Al Cabohne. Also, for developers it allows for easy implementations of these extensions.  

There are two major design decisions that contribute to modularity. First we have three modules
which are three separate Eclipse projects. One module contains basic objects and method related to
bohnanza, such as shuffling the discard pile into the draw deck. Another module implements \gls{std} 
bohnanza without extensions. The last module implements the \gls{hb} extension. All projects can
of course be tested and compiled individually. The \gls{std} and \gls{hb} extension inherit from the
base module. The base module provides an abstract class that resolves dependency resolution between
modules. Both the \gls{std} and the \gls{hb} module implement this class. Making it possible to hide
some information in modules and should result in understandable modules.
Second, we had the option to introduce classes in the the model which do have a role in the bohnanza game, such as a player area but no
real behaviour. A class representing this role would have some proxy methods for retrieving (unmodifiable) lists and other methods to
perform atomic operations. The class would have bean fields, a draw area etcetera, but not the discard pile. An atomic operation would be
to plant a card from the draw area into a bean field. An operation a player area is not capable of is to move a card from the bean field to
the discard pile. This is of course an advantage because it prevents incorrect behaviour. However we chose not to implement these kinds of
classes because it makes it hard to test and proxy methods like \texttt{Player.getPlayerArea().getFarm().getBeanField()} make the source
code unreadable. Instead we implemented methods directly in the \texttt{Player}, such that \texttt{Player.getBeanField()} is available. The
concept discussed here is closely related to the heuristic Uses Relationship one from lecture two. We find that this heuristic makes sense
but is not applicable here.

\subsection{Flexibility}
Creating flexible software is not easy. Flexibility may introduce risks in software, but it allows for faster adaptation of software when
external changes are desired. The basic bohnanza module should allow for creating an extension easily. However one should not over-engineer
software to much. That is, by trying to anticipate what a future release of the software should support. The way we support flexibility in
the bohnanza game is only for creating extensions of the \gls{std} bohnanza game. These extensions can be created by subtyping certain
classes in the base module. An alternative would be to supply hooks in our base module. But one can only
guess if this approach works well with extensions we have not implemented, such as el Cabohne. 

One can not simply override any method if behaviour needs to be changed, because this is bad practice. As
Checkstyle\footnote{http://checkstyle.sourceforge.net/} suggests a method should either be final or abstract. Therefore if one needs to
override a method a final method should be made abstract and (parts of it) should be implemented in a subtype. We find this closely
relates to the \emph{open-closed principle} of modularity. The base module is \emph{open} because
no classes are final and can thus be extended. But it is \emph{closed} -- thus harder to extend --
because methods are final. In our base module behaviour is hard to \emph{modify} but behaviour can however
be easily \emph{added}, since classes are not final. The trade-off here is that it is harder to modify behaviour of the base module and thus
one can not easily break the way we intent bohnanza to work. On the other hand, it is unforeseeable if for some extension the base module
needs to be changed. If this is the case one has to change the accessibility modifiers described earlier.
