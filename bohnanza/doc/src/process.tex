\section{development process}
First we drew a class diagram using Software Ideas Modeler\footnote{http://www.softwareideas.net/} including methods and attributes. Using
this diagram we generated Java class skeletons. Soon we came to the conclusion that many parts of our application could not be
tested well, because we tried to protect it from incorrect behaviour to much. 

% This involved creating specific instances of classes in
% constructors and hiding away to much by creating wrapper classes. One example of such a wrapper class is a \texttt{PlayerArea} which
% contains the beanfields, hand, etc, but not the discard pile. The idea is that we could use such a wrapper class to make sure some
% operations could not be done, such as moving a card from a hand to the discard pile. We did not use this approach because it was not
% testable. We removed the wrapper classes and used a dependency injection framework instead.

After standard bohnanza was implemented we began working on high bohn. We decided we could use the dependency injection framework to create
the three modules. At this moment in time we saw that in order to have both the \gls{std} and \gls{hb} implementation we needed three
instead of two modules. The main reason is that you should design for extension as suggested by Checkstyle. 
% This means one should
% explicitly make some methods abstract which should be implemented by a more concrete class. Our first intention was to just override some
% methods which needed different behaviour in high bohn than that in standard bohnanza.

After we were done with the implementations we used
UmlGraph \footnote{http://www.umlgraph.org/doc/indexw.html} to generate necessary class diagrams and write this report. 

Our design phase was rather short; we drew only a class diagram. In a bachelor course we learned proper software design with several phases
such as domain analysis, domain models, high level designs, detailed designs and more. The class diagram we drew for bohnanza is most
reasonably a detailed design. We find that the method learned during the bachelor course is well suited for creating complex software systems,
but since this approach does not fit well in the required report (it should focus on the detailed design) we decided to focus more on
explaining design decisions and applied techniques.
