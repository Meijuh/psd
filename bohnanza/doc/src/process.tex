\section{development process}
\subsection{Refactor}
\begin{itemize}
    \item First we drew a class diagram using Software Ideas
    Modeler\footnote{http://www.softwareideas.net/} including methods and attributes.
    \item Using this diagram we generated Java class skeletons.
    \item Then we tried to use a test driven development approach to test first and implement later.
    But in practice it resulted in doing both simultaneously.
    \item Soon we came to the conclusion that many parts of our application could not be tested
    well, because we tried to protect it from incorrect behaviour to much. This involved creating
    specific instances of classes in constructors and hiding away to much by creating wrapper
    classes. One example of such a wrapper class is a \texttt{PlayerArea} which contains the
    beanfields, hand, etc, but not the discard pile. The idea is that we could use such a wrapper
    class to make sure some operations could not be done. Such as moving a card from a hand to the
    discard pile. We did not use this approach because it was not testable. We removed the wrapper
    classes and used a dependency injection framework instead.
    \item After standard bohnanza was implemented we began working on high bohn. We decided we could
    use the dependency injection framework to create the three modules. At this moment in time we
    saw that in order to have both the \gls{std} and \gls{hb} implementation we needed three instead
    of two modules. The main reason is that you should design for extension as suggested by
    Checkstyle\footnote{http://checkstyle.sourceforge.net/}. This means one should explicitly make
    some methods abstract which should be implemented by a more concrete class. Our first intention
    was to just override some methods which needed different behaviour in high bohn than that in
    standard bohnanza.
    \item After we were done with the implementations we used
    UmlGraph\footnote{http://www.umlgraph.org/doc/indexw.html} to generate neccesary class diagrams.
    \item Lastly we wrote this report. 
\end{itemize}