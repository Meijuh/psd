\section{lessons learned}
Only rather late in the development process we came to the conclusion that our implementation was hard to test. Should we have used a Test
Driven Development we would have come to the conclusion to use dependency injection much earlier. So on the principle of making testable
code we might have failed, but we learned. Now we know that dependency injection is key to making testable code we do not think TDD is key
we find that sometimes implementing first and then generate a unit test to some degree can be just as convenient.

Some good principle we have learned came from using tools. After we installed moreunit it suggested to use the Mockito framework. This
allowed us to easily mock classes such as the view. Instead of subtyping a view class we could simply invoke \texttt{View view =
mock(View.class);} and \texttt{when(view.getOptionsFromHand())} \\ \texttt{.thenReturn(new ArrayList<Bean>());}. But more importantly what
we learned is that one should not do partial mocking. That is either one should mocking everything of a class or nothing. Partial mocking is
done by mocking one method and invoke the real implementation of an other method. Mockito warns the developer when he/she uses partial
mocking. Since mocking does not allow assigning values to private fields of mocked objects one has to \emph{verify} behaviour. In Mockito
this can be done with the \texttt{verify} method, e.g. \texttt{verify(discardPile, atLeastOnce()).add(new BlackEyed());}. For verifying
advanced argument captors can be used and asserting that such a captor contains a certain value.

Another good principle we have learned came from using Checkstyle. Checkstyle warned us about the fact that a method should either be final
or abstract, which we discussed earlier.

Furthermore we have learned there is only one good free and open source tool for generating class diagrams. This tool is called UMLGraph and
uses Java classes for specifying -- with visibility parameters -- certain \emph{view}s on your design. To generate these diagrams one has to
pass a custom doclet to the javadoc command. Visual Paradigm \footnote{http://www.visual-paradigm.com/} is a closed source and costly
alternative which also supports round-tripping which is very convenient for larger projects.
